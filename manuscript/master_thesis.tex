\documentclass[a4paper,10pt]{scrartcl}
%\usepackage[utf8x]{inputenc}

% Packages =========================================
\usepackage{algorithm}
\usepackage{algpseudocode}
\usepackage{amsmath}
\usepackage{amsfonts}   		% Pakete für math. Formeln, Symbole, Ausdrücke
\usepackage{amssymb}
\usepackage[toc,page]{appendix}				% more control over appendices
\usepackage[english]{babel} 				% Anpassung für mehrsprachige Dokumente
\usepackage[usenames, dvipsnames]{color}					% Nutzung von Farben
\usepackage{float}					% notw. für explizite Setzung von Grafiken
\usepackage[T1]{fontenc} 				% Zuweisung Codierschemas für Zeichensätze
\usepackage[paper=a4paper,left=40mm,right=30mm,top=20mm,bottom=25mm]{geometry} % Randmaße
\usepackage{graphicx} 					% Einbindung externer Grafiken
\usepackage{hyperref}
%\usepackage[utf8]{inputenc}				% for german characters like "ä, ö, ü"
\usepackage{listings}					% notwendig für Quellcode
\usepackage{lmodern}	 				% Moderne Version von Computer Modern
\usepackage{pdfpages}					% Einbindung einer pdf-Datei
\usepackage{pgfplots}
\usepackage{mdwlist}					% kompaktere Auflistungen
\usepackage{microtype}	 				% Optischer Randausgleich
\usepackage{natbib}
\usepackage{subcaption}					% Subfigure/-tabellen
\usepackage{scrpage2}
\usepackage{tabularx}					% Textsetzung in Tabellen in p-Spalten
\usepackage{textcomp}					% Besondere Textzeichen
\usepackage{ulem}					% Unterstreichung
\usepackage{url}					% Links
\usepackage{xfrac}					% notw. für schräggestellten Bruch
\usepackage[margin=10pt,font=scriptsize,labelfont=bf,labelsep=endash]{caption}




\begin{document}
	\thispagestyle{empty}
	\begin{center}
		\Large
		\textbf{A Neuronal Model for Visually Evoked Startle Responses in Schooling Fish}\\
		
		\vspace*{1cm}
		\normalsize
		Master thesis\\
		
		\vspace{3cm}
		by\\
		Andrej Warkentin \\
		Bernstein Center for Computational Neuroscience - Berlin
		
		\vspace{2cm}
		Supervisors:\\
		Dr. Pawel Romanczuk\\
		Bernstein Center for Computational Neuroscience - Berlin\\
		Prof. Dr. Henning Sprekeler\\
		Technische Universit\"{a}t Berlin
		
	\end{center}
	\pagebreak 
	% Title Page
	\title{masterthesis}
	
	% Inhaltsverzeichnis =====================================
	\thispagestyle{empty}
	\tableofcontents		% Inhaltsverzeichnis erstellen
	\newpage
	%\cleardoublepage	% min. eine Freiseite
	
	%% Für den Hauptteil normale Seitenzahlen
	\pagestyle{useheadings}    % wieder auf normale Seitenrahmen zurück schalten
	\pagenumbering{arabic}  % Nummerierungstyp roman | arabic
	
	\begin{abstract}
		\textbf{Abstract}\\
		Many aspects of fish school behavior can be explained qualitatively by self-propelled agent models with social interaction forces that are based on either metric or topological neighborhoods.
		Recently, startling of fish has been analyzed in its dependence of the network structure \citep{Rosenthal2015} but a mechanistic model and its influence on the collective behavior is missing.
		Here we couple a model for collective behavior with a neuronal model that receives looming visual stimulus input to initiate a startle response, inspired by the neurobiologically well-studied Mauthner cell system.
		First, we analyzed the basic properties of the startle behavior of a single fish as a reaction to a looming stimulus.
		On the group level, we looked at startling frequency as well as group cohesion and polarization depending on neuronal and collective behavior parameters via simulations of the combined model.
		Our results indicate that the startling frequency strongly depends on the dynamics of the group structure, e.g. when the group approaches a boundary of the arena.
		In summary, we took first steps towards a biologically plausible model for startle response initiation in the context of collective motion.
	\end{abstract}
	\newpage
	
	\section{Introduction}

	\section{Methods and Materials}
	
	\section{Results}
	
	\section{Discussion}
	
	\newpage
	\bibliographystyle{apalike}
	\bibliography{masterthesisbib}
	
	\newpage
	\appendix
	%\appendixheaderon
		% next line tells latex to not list sections in the Table of Contents, \protect is needed because \setcounter apparantly
		% is a fragile command (see http://www.tex.ac.uk/FAQ-protect.html)
	\addtocontents{toc}{\protect\setcounter{tocdepth}{-1}}
	\section{Appendix}

	
\end{document}          
