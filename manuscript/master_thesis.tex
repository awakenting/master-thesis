\documentclass[a4paper,10pt,hidelinks]{scrreprt}
%\usepackage[utf8x]{inputenc}

% Packages =========================================
\usepackage{algorithm}
\usepackage{algpseudocode}
\usepackage{amsmath}
\usepackage{amsfonts}   		% Pakete für math. Formeln, Symbole, Ausdrücke
\usepackage{amssymb}
\usepackage[toc,page]{appendix}				% more control over appendices
\usepackage[english]{babel} 				% Anpassung für mehrsprachige Dokumente
\usepackage[usenames, dvipsnames]{color}					% Nutzung von Farben

\usepackage{float}					% notw. für explizite Setzung von Grafiken
\usepackage[T1]{fontenc} 				% Zuweisung Codierschemas für Zeichensätze
\usepackage[paper=a4paper,left=40mm,right=30mm,top=20mm,bottom=25mm]{geometry} % Randmaße
\usepackage{graphicx} 					% Einbindung externer Grafiken
\usepackage{hyperref}
%\usepackage[utf8]{inputenc}				% for german characters like "ä, ö, ü"
%\usepackage{listings}					% notwendig für Quellcode
\usepackage{lmodern}	 				% Moderne Version von Computer Modern
\usepackage{pdfpages}					% Einbindung einer pdf-Datei
\usepackage{pgfplots}
\usepackage{mdwlist}					% kompaktere Auflistungen
\usepackage{microtype}	 				% Optischer Randausgleich
\usepackage[round]{natbib}
\usepackage{doi}
\usepackage{subcaption}					% Subfigure/-tabellen
\usepackage{scrpage2}
\usepackage{tabularx}					% Textsetzung in Tabellen in p-Spalten
\usepackage{textcomp}					% Besondere Textzeichen
\usepackage{ulem}					% Unterstreichung
\usepackage{url}					% Links
\usepackage{xfrac}					% notw. für schräggestellten Bruch
\usepackage[margin=10pt,font=scriptsize,labelfont=bf,labelsep=endash]{caption}




\begin{document}
	\thispagestyle{empty}
	\begin{center}
		\Large
		\textbf{A Neuronal Model for Visually Evoked Startle Responses in Schooling Fish}\\
		
		\vspace*{1cm}
		\normalsize
		Master thesis\\
		
		\vspace{3cm}
		by\\
		Andrej Warkentin \\
		Bernstein Center for Computational Neuroscience - Berlin
		
		\vspace{2cm}
		Supervisors:\\
		Dr. Pawel Romanczuk\\
		Bernstein Center for Computational Neuroscience - Berlin\\
		Prof. Dr. Henning Sprekeler\\
		Technische Universit\"{a}t Berlin
		
	\end{center}
	\pagebreak 
	% Title Page
	\title{masterthesis}
	
	% Inhaltsverzeichnis =====================================
	\thispagestyle{empty}
	\tableofcontents		% Inhaltsverzeichnis erstellen
	\newpage
	%\cleardoublepage	% min. eine Freiseite
	
	%% Für den Hauptteil normale Seitenzahlen
	\pagestyle{useheadings}    % wieder auf normale Seitenrahmen zurück schalten
	\pagenumbering{arabic}  % Nummerierungstyp roman | arabic
	
	\begin{abstract}
		\textbf{Abstract}\\
		Many aspects of fish school behavior can be explained qualitatively by self-propelled agent models with social interaction forces that are based on either metric or topological neighborhoods.
		Recently, startling of fish has been analyzed in its dependence of the network structure \citep{Rosenthal2015} but a mechanistic model and its influence on the collective behavior is missing.
		Here we couple a model for collective behavior with a neuronal model that receives looming visual stimulus input to initiate a startle response, inspired by the neurobiologically well-studied Mauthner cell system.
		First, we analyzed the basic properties of the startle behavior of a single fish as a reaction to a looming stimulus.
		On the group level, we looked at startling frequency as well as group cohesion and polarization depending on neuronal and collective behavior parameters via simulations of the combined model.
		Our results indicate that the startling frequency strongly depends on the dynamics of the group structure, e.g. when the group approaches a boundary of the arena.
		In summary, we took first steps towards a biologically plausible model 
		for startle response initiation in the context of collective motion.
	\end{abstract}
	\newpage
	
	\chapter{Introduction}
	A common interpretation of the function of the nervous system in animals is 
	to use the sensory input in order to make appropriate actions.
	One situation where this would be useful for the animal is the sudden 
	appearance of a predator.
	The quick response to such a sudden, unexpected stimulus is called startle 
	response and can be observed in many species \citep{Eaton1984a}.
	In fish the startle response can take the form of freezing, where the fish 
	stops moving entirely, or the form of an escape response, where it quickly 
	accelerates and moves away within less than a second.
	Escape responses in fish, also called fast starts, can be divided into the 
	three stages 1) first body bend, 2) second body bend and a third, variable 
	stage where the fish either goes into continuous swimming, coasting or 
	braking \citep{Domenici2011}.
	Due to the body shape at the end of the first stage escape responses are 
	also called C-start or S-start \citep{Domenici2011}\footnote{It should be 
	noted here that not all C-starts are escape responses because the can also 
	be involved in e.g. prey capture but we will ignore other roles in the 
	following.}.
	This thesis will focus on the C-start behavior of fish.\\
	The C-start behavior in fish has been extensively studied and one of the 
	main reasons for this is that a pair of neurons that play a major role in 
	the initiation of the C-start, have large soma and axons and are therefore 
	relatively easy to find in experiments.
	They are called Mauthner cells (M-cells), named after Ludwig Mauthner who 
	first found and described their axons \citep{Mauthner1859}.
	
	\begin{itemize}
		\begin{itemize}
		\item it can be evoked by different sensory modalities such as 
		auditory, 
		vibrational, and visual
		\item we will focus on visual modality here
		\end{itemize}
		\item the neural correlates of this behavior also have been studied 
		already
		\item before we go into the details I'll give a short overview of the 
		neuroanatomy of fish:
		\begin{itemize}
		\item structure is very similar to mammals with a spinal cord, brain 
		stem, 
		(some have cerebellum?), sensory organs and telencephalon
		\item visual system is also very similar to mammals:
		\item eyes have pupils(?), lenses, and a retina with different cones 
		and 
		rods and different types and layers of neurons with ganglion cell axons 
		building the optic nerve
		\item most fish don't have something like fovea but they do have 
		regions of 
		higher ganglion cell density \cite{Pita2015}
		\item at this point the difference to mammals/humans become bigger 
		because 
		the next station is already some sort of cortex, the optical tectum 
		(instead of suprior colliculus as in humans) (have to double-check this)
		\item from OT we have axons going to the hindbrain which is the main 
		region 
		of interest for this thesis
		\item in the hindbrain we have, among others, a pair of giant fiber 
		neurons, called Mauthner-cells
		\end{itemize}
		\item these cells have been found to play a major role in the c-start 
		behavior of fish
		\item name important examples to support this statement
		\item the mauthner cells and their surrounding circuit are highly 
		specialized, seemingly for the c-start:
		\begin{itemize}
		\item multisensory input for using all available information
		\item auditory nerve axons have mixed electrical-chemical synapses 
		directly 
		onto the lateral dendrite of the mauthner cell
		\item feedforward inhibition is stronger on contralateral side, 
		probably in 
		order to favor the ipsilateral m-cell to fire (because that would lead 
		to an escape away from the stimulus)
		\item feedback inhibition makes the m-cell only fire once because the 
		behavior is energetically expensive
		\item high input resistance plus feedforward inhibition of input for 
		high 
		threshold of activation
		\item huge axons for fast signal transmission to motor neurons in 
		spinal 
		chord
		\item in spinal chord, inhibitory interneurons seem to deactivate 
		"swimming 
		motor neurons" and activate a different population of "escape motor 
		neurons"
		\end{itemize}
		\item in first part of thesis I will study a simple neuronal model that 
		should 1) have features of the m-cell and 2) reproduce experimental 
		c-start behavior
		\item when using a neuronal model we have to decide on the level auf 
		biophysical detail that we want to capture
		\begin{itemize}
		\item the main levels to consider here are ionic currents based models 
		like 
		the hodgikn-huxley model, leaky integrate-and-fire models and rate 
		based models
		\item I chose the LIF because we are not interested in the shape of the 
		action potential but only in the timing dependent of input
		\item it also allows us to efficiently simulate it which is useful for 
		parameter explorations and fitting as well as integrating it in the 
		collective model that I will talk about next
		\end{itemize}
		\item if we think about the natural environment of fish, for many 
		species 
		this means that a single fish is surrounded by many others and that 
		they are moving more or less coordinated together as a fish shoal(less 
		ordered) or school(highly ordered)
		\item in the second part the thesis I want to understand how the 
		initiation 
		mechanism in the first part integrates into such collective behavior
		\begin{itemize}
		\item can startling also be evoked by neighboring fish that come too 
		close 
		to fast? how does this depend on the properties of the school?
		\item does the startling of a single fish spread in the school? how 
		does 
		this depend on the properties of the school?
		\end{itemize}
		\item in order to address these questions I will use an agent-based 
		model 
		for collective behavior
		\begin{itemize}
		\item it describes the collective behavior by so-called social forces 
		that 
		lead to repulsion, alignment, and attraction between fish
		\item the forces either work on neighbors in specific ranges (also 
		called 
		metric interaction) or on topological neighbors
		\item dependent on the parameters of the social forces and of the speed 
		of 
		the agents we get different modes of the collective such as highly 
		polarized and cohesive schools or a milling behavior \cite{Couzin2002}
		\end{itemize}
		\item visual ecology
		\begin{itemize}
			\item warning: making analogies to human vision is almost always misleading
			\item "sensory world of each species is unique"
			\item color vision in fish: Visual Ecology pp. 159
			\item fish orient to overhead polarization orientation in laboratory Hawryshyn 1992
			\item most fish species don't have a fovea (Encyclopedia of Fish 
			Physiology, p. 141) so that eye movements/saccades should not be 
			interpreted as fixations as is the case for humans
			\item they do have different ganglion cell densities though, see 
			\cite{Pita2015}
			\item zebrafish have a row-ordered retinal mosaic with alternating rows with LWS double cones (red and green) and rows with SWS (blue and ultraviolet)
			\item rhodopsin (based on vitamin A1, shorter, "blue" wavelengths) more in marine fish and porphyropsins (based on A2, longer, "green" wavelengths) rather in freshwater fish (more green environment)
			\item important thing to remember: the visual field above a fish is very different from the lateral view which is again different from the visual field below a fish
			\item fritsches and marshall 2002: eye movements in teleosts
			\item here I cite \cite{Tytell2008}
		\end{itemize}
	\end{itemize}
	\newpage
	\chapter{Methods and Materials}
	\section{Neuronal model}
	\begin{equation}
	I(t) = f(\theta (t))
	\label{eq:input}
	\end{equation}
	
	\begin{equation}
	\theta (t) = 2\cdot \arctan(\frac{L/2}{distance})
	\label{eq:theta}
	\end{equation}
	
	\begin{equation}
	\tau _{\rho} \frac{d\rho}{dt} = - (\rho(t) - \rho_{0}) + c_{\rho} I(t) + 
	\eta _{\rho}(t)
	\label{eq:inhib}
	\end{equation}
	
	\begin{equation}
	\tau _m \frac{dV_m}{dt} = - (V(t) - E_{L}) + R_{m} I(t) - \rho (t) +  \eta 
	_m (t)
	\label{eq:mcell}
	\end{equation}
	
	
	\section{Adiabatic approximation}
	We assume that the timescale of the Input is much higher than the timescale 
	of the dynamics of the inhibitory population so that we have a the 
	following stationary process:
	\begin{equation}
	\hat{V}_m(t) = E_{L} + I_{tot}(t) + noise
	\end{equation}
	where
	\begin{equation}
	I_{tot}(t) = R_{m} I(t) - \hat{\rho}(t)
	\end{equation}
	
	\begin{equation}
	\hat{\rho}(t) = c_{\rho} 10^{7} I(t) + \rho_{0}
	\end{equation}
	
	\begin{equation}
	I(t) = 10^{-11} c_{exc} f(\theta(t)) = 10^{-11} c_{exc} (m \cdot \theta(t) 
	+ b)
	\end{equation}
	
	We set all noise to zero and want to find the input at which the membrane 
	potential reaches the threshold $V_{t}=-61$ mV:
	\begin{equation}
	\hat{V}_m(t) \overset{!}{=} V_t
	\end{equation}
	\begin{equation}
	\Leftrightarrow E_{L} + R_{m} I(t) - c_{\rho} I(t) - \rho_{0} 
	\overset{!}{=} V_t
	\end{equation}
	Inserting values for the fixed parameters $E_{L}=-79$ mV, $R_{m}=10$ 
	M$\Omega$ 
	and $V_{t}=-61$ mV:
	\begin{equation}
	-0.079 + 10^{7} I(t) - c_{\rho} 10^{7} I(t) - \rho_{0} \overset{!}{=} -0.061
	\end{equation}
	
	\begin{equation}
	\Leftrightarrow 10^{7} I(t) - c_{\rho} 10^{7} I(t) - \rho_{0} 
	\overset{!}{=} 0.018
	\end{equation}

	\begin{equation}
	\Leftrightarrow 10^{-4} c_{exc} f(\theta(t)) (1 - c_{\rho}) - \rho_{0} 
	\overset{!}{=} 0.018
	\end{equation}
	
	\begin{equation}
	\Leftrightarrow f(\theta(t)) \overset{!}{=} \frac{180 + \rho_{0}10^{4}} 
	{c_{exc}(1 - c_{\rho})}
	\end{equation}

	\begin{equation}
	\Leftrightarrow \theta(t) \overset{!}{=} \frac{180 + \rho_{0}10^{4}} 
	{m \cdot c_{exc}(1 - c_{\rho})} - \frac{b}{m}
	\end{equation}
	\subsection{Further points}
	\begin{itemize}
		\item first paragraph
	\end{itemize}
	\newpage
	\section{Results}
	\subsection{Response properties of a single LIF neuron}
	As a first step we presented a single LIF neuron with the visual angle 
	$\theta$ over time as input current.
	In order to compare our results with experimental work (see e.g. 
	\cite{Bhattacharyya2017}, \cite{Temizer2015}, \cite{Dunn2016}) we analyzed 
	the angle, distance, latency and time-to-collision of the response.
	The response onset was defined as the time of the first spike of the LIF neuron.
	We ignore further processing time after the spike of the Mauthner cell 
	because it is in the order of milliseconds (\cite{Preuss2003}) and thus 
	irrelevant with respect to the overall response time which is in the order 
	of at least hundreds of milliseconds for visual stimuli 
	\citep{Preuss2006}.\\
	In the model, we used the basic electrophysiological parameters that were measured in larval zebrafish 4 days post-fertilization \citep{Koyama2016} and kept them fixed for all simulations.
	We analyzed the effects of parameters of a linear transformation of the input, i.e. the slope and offset and furthermore the effects of noise on the input, on the initial condition, and on the spiking threshold.
	%TODO: add variable names
	%TODO: add formulas
	All parameters are listed in table \ref{tab:neuroparams}.\\
	effects:
	\begin{itemize}
		\item effects of increasing m:
		\begin{itemize}
			\item mean response distance: mean increases linearly independent 
			of threshold noise (only for high threshold noise slightly 
			sub-linear)
			\item variance of response distance: increases linearly for small 
			threshold noise (except for a high lv value and low threshold 
			noise, this is due to a very low mean and outliers that distort the 
			standard deviation estimate), increases sub-linearly for medium 
			threshold noise, slightly decreases for high threshold noise
			\item mean response angle: decreases exponentially  independent of 
			threshold noise
			\item variance of response angle: slightly decreases independent of 
			threshold noise
			\item mean time to collision: absolute value increases linearly 
			independent of threshold noise, decreases more strongly for higher 
			L/V values
			\item variance of time to collision: very small increases for L/V 
			values smaller than 0.9, for L/V values above 0.9 the variance is 
			in general higher, for small threshold noise it is smallest for 
			medium m-values and for higher threshold noise it also increases 
			with m
			\item mean response time: very similar to TTC
		\end{itemize}
		\item effects of increasing threshold noise:
		\begin{itemize}
			\item mean response distance: 
		\end{itemize}
	\end{itemize}
	\begin{table} [!th]
		\begin{center}
			\begin{tabular}{|l|c|p{7cm}|}
				\hline
				\textbf{Parameter} & \textbf{Value (unit)} & \textbf{Comment} \\
				\hline
				$E_L$ & -79 mV & Resting potential\\
				$R_M$ & 10 MOhm & Membrane resistance\\
				$\tau_{m}$ & 23 ms & Membrane time constant\\
				$V_t$ & -61 mV & Mean spiking threshold\\
				$dt$ & 0.001 s & Integration time step\\
				$T$ & 5 s & Total time\\
				$sd_{thr}$ & 1 mV & Standard deviation of spiking threshold noise\\
				$sd_{I}$ & 5 mV & Standard deviation of input noise\\
				$sd_{init}$ & 1 mV & Standard deviation of initial condition noise\\
				$m$ & 1 \textdegree/s  & Slope of linear transformation\\
				$b$ & 0 \textdegree & Offset of linear transformation\\
				\hline
			\end{tabular}
		\end{center}
		\caption{Parameters of the single LIF neuron model with a looming stimulus input. Parameters that were explored are indicated either by a value range such as e.g. for $\mu_s$ or by a set with all explored values inside of curly brackets such as e.g. for $\sigma_s$.}
		\label{tab:neuroparams}
	\end{table}
	\begin{itemize}
		\item Effect of input transformation
		\item Effect of different noise sources
		\item Effect of input type
	\end{itemize}
	\subsection{Input}
	\begin{itemize}
		\item 
	\end{itemize}
	\subsection{Feedforward inhibition}
	\begin{figure}[H]
		\centering
		\includegraphics[width=0.6\textwidth, 
		height=0.4\textheight]{../figures/isaacson2001_figure4.jpg}
		\caption{how input sharpens tuning. From \cite{Isaacson2011}}
		\label{fig:feedf}
	\end{figure}
	\begin{itemize}
		\item 
	\end{itemize}
	\subsection{Cross-inhibition}
	\begin{itemize}
		\item 
	\end{itemize}
	\subsection{Feedback inhibition}
	\begin{itemize}
		\item 
	\end{itemize}
	\chapter{Discussion}
	\begin{itemize}
		\item we focus here on the experimental results from 
		\cite{Bhattacharyya2017} but one should keep in mind that their results 
		might be specific to properties of experiment such as fish handling, 
		fish age, species, arena, environment, stimulus setup (projection on 
		screen)
	\end{itemize}
	\newpage
	\bibliographystyle{abbrvnat}
	\bibliography{masterthesisbib}
	
	\newpage
	\appendix
	%\appendixheaderon
		% next line tells latex to not list sections in the Table of Contents, \protect is needed because \setcounter apparantly
		% is a fragile command (see http://www.tex.ac.uk/FAQ-protect.html)
	\addtocontents{toc}{\protect\setcounter{tocdepth}{-1}}
	\section{Appendix}

	
\end{document}          
