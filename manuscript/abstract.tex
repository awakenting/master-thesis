\chapter*{Abstract}
\addchaptertocentry{Abstract}
		Many aspects of fish school behavior can be explained qualitatively by self-propelled agent models with social interaction forces that are based on either metric or topological neighborhoods.
		Recently, startling of fish has been analyzed in its dependence of the network structure 
		\citep{Rosenthal2015} but a mechanistic model and its influence on the collective behavior 
		is missing.
		Here we coupled a model for collective behavior with a neuronal model that receives looming visual stimulus input to initiate a startle response, inspired by the neurobiologically well-studied Mauthner cell system.
		First, we analyzed the basic properties of the startle behavior of a single fish as a reaction to a looming stimulus and built a neuronal model to reproduce the startle behavior.
        Next, we fitted the neuronal model to experimental, behavioral data from larval zebrafish.
		On the group level, we included the fitted neuronal model in the collective behavior model and looked at startling frequency as well as group cohesion and polarization depending on collective behavior parameters via simulations of the combined model.
		Our results indicate that the fitted neuronal model can lead to experimentally observed startling frequencies and that there are nontrivial relationships between the startling frequency and the group order so that more investigation is needed.
		In summary, we took first steps towards a biologically plausible model 
		for startle response initiation in the context of collective behavior.

\chapter*{Zusammenfasssung}

Viele Aspekte des Verhaltens von Fischschulen k\"onnen qualitativ durch Modelle erkl\"art werden, die auf selbst-angetriebenen Agenten basieren, die durch soziale Kr\"afte interagieren, die sich entweder aus metrischen oder topologischen Nachbarschaften ergeben.
Vor kurzem wurde das Schreckverhalten von Fischen auf seine Abh\"angigkeit von der Netzwerkstruktur hin untersucht \citep{Rosenthal2015} aber ein mechanistisches Modell und der Einfluss eines solchen Modells auf das kollektive Verhalten fehlt noch.
Hier haben wir ein Modell f\"ur Kollektivverhalten mit einem neuronalen Modell gekoppelt, das visuellen Input bekommt um dann das Schreckverhalten auszul\"osen, was durch das neurobiologisch bereits gut untersuchte Mauthner Zellsystem inspiriert ist.
Zun\"achst haben wir die Grundeigenschaften des Schreckverhalten bei einzelnen Fischen untersucht, die mit einem drohend auftauchendem Stimulus konfrontiert werden und ein neuronales Modell aufgestellt, dass dieses Verhalten reproduzieren kann.
Danach haben wir das neuronale Modell auf experimentelle Verhaltensdaten von Zebrafischen gefittet.
Auf der Gruppenebene haben wir das gefittete neuronale Modell in das Modell f\"ur Kollektivverhalten integriert und mittels Simulationen die H\"aufigkeit des Schreckverhaltens sowie die Gruppenkoh\"arenz und die Gruppenpolarisation in Abh\"angigkeit von Parametern des kollektiven Modell untersucht.
Unsere Ergebnisse weisen darauf hin, dass das gefittete Modell zu experimentell beobachteten H\"aufigkeiten des Schreckverhaltens f\"uhren kann und dass es nicht-triviale Zusammenh\"ange zwischen der H\"aufigkeit des Schreckverhaltens und der Gruppenstruktur gibt, die weiterer Untersuchungen bed\"urfen.
Zusammenfassend l\"asst sich sagen, dass wir erste Schritte in Richtung eines biologisch, plausiblen Modell f\"ur die Initierung von Schreckverhalten im Kontext von kollektivem Verhalten machen konnten.