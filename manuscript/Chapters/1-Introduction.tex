\chapter{Introduction} % Main chapter title
	A common interpretation of the function of the nervous system in animals is to use the sensory 
	input in order to make appropriate actions.
	One situation where this would be useful for the animal is the sudden appearance of a predator.
	The quick response to such a sudden, unexpected stimulus is called startle response and can be 
	observed in many species \citep{Eaton1984a}.
	In fish, the startle response can take the form of freezing, where the fish	stops moving 
	entirely, or the form of an escape response, where it quickly accelerates and moves away within 
	less than a second.
	Escape responses in fish, also called fast starts, can be divided into the three stages 1) 
	first body bend, 2) second body bend and a third, variable stage where the fish either goes 
	into continuous swimming, coasting or braking \citep{Domenici2011}.
	Due to the body shape at the end of the first stage escape responses are also called C-start or 
	S-start \citep{Domenici2011}\footnote{It should be noted here that not all C-starts are escape 
	responses because they can also be involved in e.g. prey capture but we will ignore other roles 
	in the following.}.
    C-starts can be further divided into short-latency (SLC) and long-latency C-starts (LLC) (\cite{Burgess2007}, \cite{Domenici2010}).
	This thesis will focus on the C-start behavior of fish.\\
	The C-start behavior in fish has been extensively studied and one of the main reasons for this 
	is that a pair of neurons that play a major role in the initiation of the C-start, have large 
	soma and axons and are therefore relatively easy to find in experiments.
	They are called Mauthner cells (M-cells), named after Ludwig Mauthner who first found and 
	described their axons \citep{Mauthner1859}.\\
	Before going into the details of the M-cell circuit we will give a brief overview of the 
	nervous system of fish to provide some context.
	Since we will later focus on visually evoked C-starts we will go into more detail when it comes 
	to the visual pathways in the brain.
    For all of the following, one should keep in mind that there is a great variety of fish species that live in very different environments so that one can expect large deviations from the description below when looking at a single species.\\
	The overall structure of the central nervous system of fish is very similar to mammals.
	Starting at the caudal end there is the spinal cord with descending motor pathways and the 
	ascending sensory pathways.
	The spinal cord goes over into the hindbrain region with the medulla and the cerebellum.
	This is followed by the midbrain which comprises the rostral part of the brainstem and a roof 
	region, the tectum.
	The remaining forebrain consists of the diencephalon and the telencephalon \citep{Butler2011}.
	In terms of sensory organs fish are equipped with the same senses as mammals and additionally 
	have the lateral line organ that senses lower frequency signals around the body such as e.g. 
	water flow and in some cases organs that can sense electrical fields.\\
    The visual system and the eyes in particular are again similar to mammals.
	Going from outside to inside, the eyes consist of the cornea, the lens surrounded by the iris and the retina followed by the photoreceptors which build the most inside layer \citep{Kroeger2011}.
    There is additionally aqueous humor between cornea and lens and a more viscous, almost solid humor between lens and retina.
	In contrast to mammals the pupils of fish are not responsive to the amount of light in the 
	environment.
	Fish mostly have rods and three different cones although across species there are up to seven 
	different types of cones \citep[Chapter~7]{Cronin2017}.
	The retina has different types of neurons that build different layers.
	The output neurons of the retina are the ganglion cells which show different kinds of tuning \citep{Antinucci2018}
	and whose axons build the optic nerve when they exit the eye \citep{Levine2011}.
	Although most fish don't have a fovea as we know it from humans there are retinal areas of 
	higher ganglion cell and photoreceptor densities \citep{Pita2015}.
	Most of the ganglion cell axons cross sides and end up in the optic tectum which is the 
	homolog of the mammalian superior colliculus.
	While in humans the main path of visual processing is thought to go through the primary visual cortex and after that to higher visual areas, in fish the optic tectum seems to be the main site of processing of visual information.
	Similar to cortical areas it is comprised of different layers, also receiving input from other 
	senses and other brain areas such as the telencephalon.
	The output of the optic tectum goes, among other regions, to the reticular nuclei in the 
	hindbrain where we also find the Mauthner cell and can thus come the M-cell circuit.\\
	The M-cell is located in the hindbrain and has two major dendritic branches, the ventral 
	dendrite and the lateral dendrite.
	It receives multisensory input which is divided between the two dendritic branches.
	The lateral dendrite receives auditory and lateral line input whereas the ventral dendrite 
	receives visual input via the optic tectum mentioned before.
	While this means that the visual input is highly processed by the neuronal networks in retina and optic 
	tectum before it arrives at the M-cell the auditory input comes directly from the auditory 
	nerve.
	This might be one of the reasons why the physiology of the auditory input has been studied in 
	more detail than the visual part.
	We will therefore continue to describe the properties of the auditory processing and will have 
	to assume that they also hold for the processing of the visual input.
	The synapses between auditory nerve and lateral dendrite are called club endings and transmit 
	auditory signals via electrical as well as chemical mechanisms which leads to Excitatory 
	Post-Synaptic Potentials (EPSPs) that consist of a fast and a slow component \citep{Korn2005}.
	At the same time the auditory nerve excites an interneuron which itself inhibits the M-cell.
	One interpretation of this feed-forward inhibition (FFI) is to increase the threshold for 
	initiating the startle response.
	But this is not the only function of these interneurons because they also inhibit the 
	contralateral M-cell as well as their contralateral counterparts \citep{Koyama2016}.\\
	As \cite{Koyama2016} conclude, this microcircuit is probably responsible for the decision of which direction to escape to.
	To illustrate why this connectivity makes sense in this decision-making context, let us 
	consider an auditory stimulus coming from the left side:
	It will inhibit the M-cell on the right side, inhibit the interneurons on the right side, excite the M-cell on the left side and also inhibit it.
	Effectively, we have an increased inhibition of the right M-cell and an increased excitation of the left M-cell.
	Because the axons of the M-cells cross sides, an action potential of the M-cell on the left 
	side will lead to a contraction of muscles on the right side, resulting in movements of head 
	and tail away from the stimulus on the left side.\\
	There are further properties that seem to make the M-cell specialized for initiating the 
	C-start.
	Additional to the feed-forward inhibition the big size of the some of the M-cell leads to a 
	high input resistance which again increases the threshold for incoming currents to initiate an 
	action potential.
	The axon of the M-cell is unusually big as well which results in a fast signal transmission 
	when an action potential is initiated and thus allows for fast reactions.
	The axon is also connected to so-called Cranial Relay Neurons (CRN) that putatively excite interneurons that provide feedback inhibition to both M-cells (\cite{Koyama2011}, \cite{Hale2016}).
	This is thought to prevent repetitive firing of the M-cell that fired in the first place and 
	also to prevent the contralateral M-cell to fire shortly after.
	Apart from this feedback inhibition the axon goes through the whole spinal chord with 
	colaterals that go to the motor neurons on the contralateral side.
	And also at this level we have again interneurons that putatively have the role of inhibiting a 
	different set of motor neurons that are responsible for steady swimming movements 
	\citep{Song2015}.\\
	The exact role of the M-cells and the surrounding circuit in the C-start behavior is still a 
	subject of study.
	The current state of research suggests that the M-cell is indispensable and inducing for the 
	the first phase of short-latency C-starts.
	Nevertheless, if the M-cell is ablated the fish are still able to perform long-latency C-starts 
	(\cite{Lacoste2015}, \cite{Dunn2016}).
	Furthermore, there is another population of neurons, the Spiral Fiber Neurons (SFNs), that, if ablated, increase the latency of C-starts in a similar manner as the ablation of the M-cell does \citep{Lacoste2015}.
    Their axons are targeting the initial axon segment of the M-cell but their input is so far unknown to the best of our knowledge.\\
	In the first part of this thesis we aim to make first steps towards a mechanistic understanding 
	of functional role of the M-cell circuit for the C-start behavior.
	For this, we will greatly simplify the physiological properties of the M-cell and use a Leaky 
	Integrate-and-Fire (LIF) model to capture the relevant dynamics.
	We did not choose the supposedly more realistic\footnote{Although, \cite{Brette2015} argues that Integrate-and-Fire models are more realistic if one looks at spike initiation and only compares single-compartment models.}  Hodgkin-Huxley-like model type that takes into account different ion currents because we were not interested in the action potential shape but rather in the action potential timing dependent on the input.
	The simpler LIF model also allowed for more efficient simulation which was also useful for the 
	integration of the neuronal model in the collective behavior model in the second part.\\
	For the input of the M-cell we will assume that the visual input, coming from the optic tectum, 
	is the result of a feature extraction of the visual scene.
    Evidence, that the startle response can be evoked visually has been shown in several experiments that use a visual looming stimulus and which we will describe in more detail in Chapter \ref{ch:expm}.
	Taken together, this will allow us to link parameters of the neuronal model to behavioral 
	response properties and to fit the model to experimental data.\\
	%TODO: describe previous work in "modeling the mauthner system"
	While the first part of the thesis is concerned with the behavior of single fish, for many fish species the natural environment is more likely to live in a group together with many other fish.
	Such groups of fish that move around together are called \textit{shoals} if they are rather uncoordinated and \textit{schools} if they move in a highly ordered manner.
    How fish schools achieve this high degree of coordination even during complex maneuvers e.g. when a predator attacks has been of great interest.
    In a previous study, \cite{Rosenthal2015} analyzed how the spontaneous startling of a single fish would spread in the school and found that the local structure of the visual network is predictive of how much the startling would spread.
    In this context we were interested how our fitted model for the visual initiation of a startle response would fit into the collective behavior.
	In more detail, we were interested in the following questions:
	Can the startle response be evoked spontaneously in the fish school e.g. when neighboring fish 
	come too close too fast?
    Is the probability to startle dependent on the position and orientation of the fish within the school?
	Does the startling of a single fish spread in the school?
	How do these effects depend on the properties of the school?\\
	In order to address these questions we used an agent-based model that describes the 
	interactions between fish by so-called social forces.
	Typically, there are the three forces repulsion, alignment and attraction and they can work 
	either on neighbors within a specific range (metric interaction) or only on topological 
	neighbors (topological interaction).
	An example for a topological type of interaction would be to only consider the neighboring 
	cells of the Voronoi tessellation of the group of fish.
	Using such an agent-based collective behavior model with metric interaction, \cite{Couzin2002} 
	could show that the simulated fish school shows different modes of behavior dependent on the 
	parameters of the social forces.
	While in one mode the collective would be uncoordinated but stay loosely together, it would 
	move highly polarized and cohesive in another parameter regime or show a kind of milling 
	behavior in a third mode.\\
    In this work we added on top of the social forces the fitted neuronal model from the first part so that if the model M-cell of an agent reaches its threshold during the simulation, the agent performs a stereotyped startle response.
    In this situation we don't have a single looming stimulus that we could directly transform into the visual input for each agent but instead the visual field is much more dynamic, consisting of many neighboring agents at different distances.
    Therefore, we analyzed and compared several ways in which an agent could in principal process its visual field resulting in different visual inputs for the neuronal model.\\
    To summarize, in the next chapter we will formulate the neuronal model followed by a chapter where we describe the experiments that we want to reproduce and how we fit the model to the experimental data.
    We continue with a description of the model for the collective behavior and how we integrated the neuronal model.
    This is followed by an analysis of the startling patterns in the collective behavior and finally, we discuss the overall results.